\section{Current Progress}
\label{sec:currentprogress}
Thus far, a pre-study and the first study had been conducted. The pre-study investigated the design space, an extension of the Ethereal Planes framework \cite{Ens2014a}, while the subsequent study used the feedback of the investigation to test different layouts in a simulated task.

\begin{itemize}
	\item Pre-study: The goal of this study was to identify which factors in the design space had the most impact on users. The study investigated four categories:
		\begin{itemize}
			\item Visual Adaptivity Preference: The users' preference of the placement of content when in an environment that is visually busy or visually flat.
			\item Window Transform Behaviours: The users' preference in the movement of windows based on the position of the head. 
			\item Spatial Region Preference: The users' preferred regions where they would like their content to be displayed displayed.
			\item Window Organization: The users' preferred location for the content in the specific regions noted in the Spatial Region Preference.
		\end{itemize}
	  	The context of the procedure being explored was neonatal resuscitation. We collected feedback from four experts with years of experience with the procedure, and recorded their experience with the Hololens.
	 \item Study 1: Using results from the feedback of the pre-study, a study protocol was created to test the efficiency of different layouts. The participants were asked to use the information streamed through the Hololens in order to complete the tasks and objectives of a simple maze game. There were three layouts tested in this study, two were created based on previous research \cite{Ens2014, Ens2015}, and one was created based on the guidelines extrapolated from the pre-study. This study was conducted with twelve participants (M=8:F=4), the data was analyzed, and no significant effect was found (p $>$ .05). In order to improve the experimental procedure, adjustments based on the feedback obtained is required.
\end{itemize}