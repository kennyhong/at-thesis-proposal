\section{Problem Statement}
\label{sec:problemstatement}
Only in the past few years have we seen advancements in the technology enabling us to effectively implement this type of research. With new head-worn displays (HWD) such as the Hololens \cite{Hololens}, we are given the opportunity to create spaces using the design framework \cite{Ens2014a} or implementing the spaces in order to test the effectiveness of them while performing in-situation tasks.
In the field of medicine, augmented reality applications (ARAs) are used to help educate medical practitioners in various techniques: These applications are used to support their training and provide them with a way to practice with simulated experiences. \cite{Barsom2016} reviewed the validity and the effectiveness of using ARAs in medical training. They investigated various applications, from giant workstations to virtual reality, and found that mixed reality can bring digital information into the real world. Using the HWDs as an assistive tool in procedural tasks is a novel topic in HCI, and can be beneficial in various fields (e.g., medical). Thus far, researchers have focused on designing and developing mixed reality interfaces. However, perhaps due to hardware limitations, there is no research testing these interfaces while performing a complicated task on a dedicated HWD thus far, despite the prospective benefits. Therefore, we propose to study the effectiveness and the potential benefits of performing a procedural task while receiving supplementary information from a dedicated HWD. 
In this work, I will be addressing three issues in order to optimize the placement of the supplementary information given the space around the user. Based on the user's environment, I will be investigating:
\begin{enumerate}
	\item How the information should be positioned around the user.
	\item How the information should adapt to visually 'cluttered' and 'busy' spaces within the user's viewpoint.
	\item Finally, I will be looking at how we can provide the proper level of information in order to complete a task.
\end{enumerate}