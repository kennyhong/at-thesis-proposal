\section{Introduction}
\label{sec:introduction}
I propose to investigate the potential outcomes of providing supplementary information on a mixed reality device while performing an intricate procedural task. I plan to explore if there is a significant difference between using a mixed reality head worn display to provide the information and the current conventional ways, such as using posters, charts, and other physical mediums. Procedural tasks require a set of step by step instructions in general. If the task at hand is complex, the instructions need to be simple in order to reduce the cognitive load. For example, medical procedures often require the task to be completed piecemeal, so the task can be completed successfully and importantly, in a timely manner. Indeed, various medical procedures often consist of numerous convoluted steps, and thus challenging. I believe that having new technology, which allows users to view informative data while they are executing a procedural task, will be beneficial. By using a modern wearable mixed reality device such as the Hololens, various types of useful/important information can be provided consistently across multiple areas in an assistive information space: We are able to leverage the new head-worn display's spatial mapping and positional tracking technology to maintain consistent positioning of the information space layout. By providing the user essential information concerning their task (e.g., instruction) and supplementary data in an adaptive information space, I hope to reduce the users' cognitive load to facilitate their performance.