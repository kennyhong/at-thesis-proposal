\section{Evaluation}
\label{sec:evaluation}
This research will be evaluated based on three studies:
\begin{enumerate}
\item The first study will determine the design space for the application. Based on professional interviews and questionnaires, a few virtual layouts will be discussed and implemented for a study that determines how the users interact with the different virtual layouts in a within-subject study where each participant will be testing all three layouts. I will be observing the effects of each layout and answering questions such as: Which had the greatest error rate? Which was the most accurate? In this study, qualitative feedback will also be obtain from the participants and looked at as well. An Analysis of Variance (ANOVA) will be used to determine the effects of each virtual layout.

\item Based on the results of study 1, the experiment will be modified to test the dynamicity of the information spaces. Using the layout(s) which show significant effect, I will be evaluating the effectiveness of mesh-aware information spaces and the space's ability to adapt to the user's head position. This study will determine the effectiveness of having the virtual 2D windows adapt to certain real-world spaces such as walls, tables, and other objects. Simply put, in this study, I would like to observe the effects having the information space adapt to visually busy areas and see how variables such as the direction that the information adapts to (in front, behind, up, down, right left of the busy area) effect the same variables tested above.

\item Finally based on both studies, I hope to create an experiment involving the primary users of the design framework in order to validate the results of both studies. I hope to combine aspects of study one and two in order to create an experience that would prove the results of both studies and in turn, validate the guidelines created.
\end{enumerate}